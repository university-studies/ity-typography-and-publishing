%Encoding: utf-8
%Projekt 4. do predmetu ITY
%Autor: Pavol Loffay, 1. rocnik BIB-30, xloffa00@stud.fit.vutbr.cz
%Datum: 21.4.2011

\documentclass[11pt,a4paper,titlepage]{article}


\usepackage[czech]{babel}
\usepackage[top=2.5cm, left=1.5cm, text={18cm, 25cm}]{geometry}
\usepackage[IL2]{fontenc}
\usepackage[utf8]{inputenc}

%ceske uvodzovky
\newcommand\czuv[1]{\quotedblbase #1\textquotedblleft}
%url pre cesku normu
\usepackage{url}
\DeclareUrlCommand\url{\def\UrlLeft{<}\def\UrlRight{>} \urlstyle{tt}}

%textova cast
\begin{document}
%uvodna strana
\begin{titlepage}
	\begin{center}
		{\Large \textsc{Fakulta informačních technologií}}\\
		{\Large \textsc{Vysoké učení technické v~Brně}}\\
		\vspace{\stretch{0.382}}
		\bigskip
		{\LARGE Typografie a publikování\,--\,4. projekt\\}
		\medskip
		{\Huge Bibliografické citace}
		%\bigskip
		\vspace{\stretch{0.618}}
	\end{center}
	\Large{\today \hfill Pavol Loffay}
	\bigskip
\end{titlepage}


\section{Typografia}
	Systém \LaTeX\ vám umožní vytvárať profesionálne vysádzané dokumenty viz \cite{latex:rybicka}.
	Pred začatím práce s systémom je vhodné si nainštalovať všetky balíčky, 
	ktoré začínajú slovom \emph{tetex} viz \cite{michal:svamberg:root}. Pričom niektoré 
	funcie \LaTeX u neodpovedajú českej norme: \textit{\czuv{standardní BibTeXové styly nejsou přeloženy 
	do češtiny a formátování položek nesouhlasí s normou}} \cite{david:martinek:latex}.
	Pokiaľ sa rozhodnete vytvoriť prezentáciu je vlhodné použiť rozširujúcu triedu \emph{Beamer}, ktorá 
	spôsobí, že príprava bude shodná s prípravou bežného dokumentu, viz \cite{petr:zelenka:abclinux}.
	Sázdanie matematiky je veľmi veľkou silou systému \LaTeX\ viz \cite{latex:math}.
	Pri správnom použití odsadzovania a členenia dokumentu môžeme dosiahnuť, že aj zdrojový súbor 
	bude veľmi prehľadný. Existuje mnoho editorov pre pohodlnú prácu  viz \cite{latex:bakalarka}.
	Systém \LaTeX\ je svojou syntaxov a členitosťou podobný programovacím jazykom viz \cite{latex:diplomovka}.
	
	\noindent Ďalej by som spomenul, že kozmický program z NASA je ohrozený viz \cite{cio:clanok}.
	V dnešnom svete IT je veľmi ťažké sa presadiť, hlavne na svetovej úrovni pretože trh je
	preplnený viz. \cite{cio:casopis}. Keď už hovorím o presadení sa v spoločnosti na globálnej úrovni je
	nutné spomenúť, že plynulé ovládanie alglického jazyka je až nutné viz \cite{typo:clanok}.

\newpage
\bibliographystyle{czechiso}
\bibliography{citacie}

\end{document}